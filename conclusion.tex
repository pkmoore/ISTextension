\section{Conclusion}
\label{SEC:conclusion}

When an application fails upon deployment,
the cause could likely be the result
of unexpected interactions between
an application and its environment.  Although finding and eliminating
faults in applications is a key concern for software developers, it is
impractical to run actual tests  of an application in every possible
environment in
which it could be deployed.  To address this problem, we introduced
CrashSimulator, a strategy and tool
that can determine whether an application will
respond correctly to anomalous environmental conditions.

The tool records system call traces from the application under test,
mutates return values and/or program state to simulate execution in the
anomalous environment, and uses sub- sequent system calls to decide whether
a correct or incorrect response as occurred. This allows CrashSimulator to
test an application running in one environment as if it were running in
another. Operating on system calls gives the tool  a ``universal'' way to
encode and inject anomalies. Consequently, a set of mutations can be
collected from any existing application for use in testing other new or
existing applications. In this way, an ever-expanding ``test suite'' can be
created, allowing lessons learned from bugs in one application to benefit
many others.

Our evaluation of CrashSimulator has shown it to be both usable and
effective.  In our survey of developers, CrashSimulator compared favorably
against both AFL and Mutiny across self-reported developer skill levels.
CrashSimulator was particularly favored by developers with more operating
systems experience, but even developers with low OS experience were pleased
with its ability to let them find bugs by taking advantage of
the expert knowledge encoded in the tool.
In our user study, a total of
19 new bugs were identified in popular applications.
These bugs have been reported to the
affected parties and 5 have already been corrected.

%  CrashSimulator was able to identify bugs related to unusual
%  file types in 15 applications and bugs related to slow network performance
%  in 10 network applications and libraries.  Additionally CrashSimulator
%  found filesystem related bugs in 6 applications and libraries with
%  facilities for moving files within a system's filesystem.  The low overhead
%  introduced by CrashSimulator's technique meant that it was able to find
%  these bugs quickly in spite of its unoptimized implementation.

Future work will include expanding the repository of anomalies and their
checkers, as well as exploring opportunities to further automate the
discovery process.  In the long term, we envision a
public repository of anomalies along with CrashSimulator test patterns that
can be applied to new or existing applications.
