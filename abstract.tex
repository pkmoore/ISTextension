\begin{abstract}

A common problem for software developers is that applications tend to
exhibit new bugs \textit{after} they have been deployed.
To this point,
preventing bugs related to
network,
operating system,
file system,
and other similar environmental features
has proven difficult
largely because there is no way to
know how an application will react to every environment without
time-consuming and expensive testing.
Enter CrashSimulator, a tool that uses
information about how environmental differences
affect applications
to identify incorrect behavior \textit{before} deployment.
The tool utilizes \textit{Anomalous Environment Simulation},
a novel technique based
on the key insight that unusual environmental properties can
be transformed into a set modified responses
the application would receive
were it interacting with the environment in question.
The tool is able
simulate these modified responses
to impose anomalous environmental conditions
on the application.
If the application does not respond correctly to a given scenario,
the tool reports that a bug may exist.
We evaluated CrashSimulator against a set of the most popular
Linux applications selected
from the Coreutils project and the Debian popularity contest.
Our tests found a total of XXXX bugs in YYYY applications,
the consequences of which range from hangs and crashes, to data loss and
remote denial of service conditions.

\end{abstract}
