\begin{abstract}

A common problem for software developers is that applications tend to
exhibit new bugs \textit{after} they have been deployed.
Preventing bugs related to
network,
operating system,
file system,
and other similar environmental features
has proven difficult in the past
largely because there is no way to
know how an application will react to every environment without
time-consuming and expensive testing.
Enter CrashSimulator, a tool that utilizes
evidence of an application's failure
in a given environment
as a predictor of the same inadequacy in other applications.
The key to the tool
is a technique called
\textit{Anomalous Environment Simulation},
which allows
unusual environmental properties
found in the interactions between an application
and its environment to be
be transformed into a set of responses
that recreate the environment in question.
Simulating these responses
on a running application
imposes anomalous environmental conditions
on it.
If the application does not respond correctly to a given scenario,
the tool reports that a bug may exist.
Over time,
the tool's library of modified responses can be expanded
allowing developers to learn of potential bugs
in new applications
before deployment.
To test the tool,
we evaluated CrashSimulator against a set of the most popular
Linux applications selected
from the Coreutils project and the Debian popularity contest.
Our tests found a total of XXXX bugs in YYYY applications,
the consequences of which range from hangs and crashes, to data loss and
remote denial of service conditions.

\end{abstract}
