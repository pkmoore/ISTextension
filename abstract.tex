\begin{abstract}

A common problem for software developers is that applications exhibit new
bugs after they have been deployed
and, to this point,
avoiding these bugs has proven difficult.
Enter CrashSimulator, a tool that uses information about how environments
differ to identify incorrect behavior before the application is
deployed.  The tool modifies system call return values and side effects
to simulate an anomalous environment and makes a determination about
whether or not the application is responding correctly to it.  As
unusual environmental properties are encoded as differences in system
call behavior, CrashSimulator can use incorrect behavior identified in
one application running in one environment to determine whether other
applications would make the same mistake in that environment.

We evaluated CrashSimulator against set of the most popular
Linux applications selected
from the Coreutils project and the Debian popularity contest.
Our tests found a total of was XXXX bugs YYYY applications,
the consequences of which range from hangs and crashes to data loss and
remote denial of service conditions.  In addition, we conducted
a user study in which
ZZZ participants used both CrashSimulator and two similar tools (AFL
and Mutiny) to hunt for bugs in popular applications and found the tool
was relatively easy to use, even for developers with limited experience
in the underlying operating systems concepts.
This study revealed
that CrashSimulator's strategy of identifying bugs
is effective, and yielded results for
developers with diverse backgrounds, and across skill sets.

\end{abstract}
