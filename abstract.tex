\begin{abstract}

A common problem for software developers is that applications tend to
exhibit new bugs \textit{after} they have been deployed.
To this point,
preventing these types of bugs has proven difficult
largely because there is no way to
know how an application will react to every environment without
time-consuming and expensive testing.
Enter CrashSimulator, a tool that uses information about how environments
differ to identify incorrect behavior \textit{before} an application is
deployed.  The tool utilizes \textit{Results Based Simulation},
a novel technique based
on the key insight that unusual environmental properties can
be encoded as a set of modifications to
the results received
when an application communicates with its environment.
The tool is able
to impose anomalous environmental conditions
on another application
by applying these modifications.
If the application does not make an effort to respond to a given scenario,
the tool reports that a bug may exist.

We evaluated CrashSimulator against a set of the most popular
Linux applications selected
from the Coreutils project and the Debian popularity contest.
Our tests found a total of XXXX bugs in YYYY applications,
the consequences of which range from hangs and crashes, to data loss and
remote denial of service conditions.  In addition, we conducted
a user study in which
ZZZ participants tested real world applicaitons using CrashSimulator.
This effort resulted in the identification of an additional WWWW new
bugs.
Our combined results demonstrate
that CrashSimulator's strategy of identifying bugs
is effective, and could yield results for
developers with diverse backgrounds, and across skill sets.

\end{abstract}
