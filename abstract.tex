\begin{abstract}

A common problem for software developers is that applications exhibit new
bugs resultant from \emph{environmental} differences after they have been
deployed.  These environmental differences, such as slightly different
behavior due to different OS versions, file systems, network
middleboxes, etc. often manifest themselves as failures or crashes of
the application.  These costly encounters are difficult to avoid
because it is not feasible to test all of an application's potential
environments.

CrashSimulator is a technique and tool that uses information about how
environments differ to identify incorrect application behavior before
the application is deployed.  The tool replays a recording of the
application under test as a means to monitor and control execution.
During replay, CrashSimulator modifies system call return values and
side effects to simulate an anomalous environment. It then makes a
determination about whether or not the application is responding to it.
Encoding unusual environmental conditions as differences in system call
behavior means CrashSimulator can use incorrect behavior identified in
one application running in one environment to determine whether other
applications would make the same mistake in that environment.

We evaluated CrashSimulator by using it to test a set of the most popular
Linux applications selected
from the Coreutils project and the Debian popularity contest. The
result of this evaluation was 84 bugs identified in 31 applications
with consequences ranging from hangs and crashes to data loss and
remote denial of service conditions.  We also compared CrashSimulator
to two similar tools (AFL and Mutiny) in a
study in which ZZZ participants used the tools to
hunt for bugs in popular applications.  This study revealed
that CrashSimulator's strategy of identifying bugs
is effective, and yielded results for
developers with diverse backgrounds, and across skill sets.
As YYY novel bugs were identified and reported during the study,
CrashSimulator is poised to help uncover a
great deal of previously unidentified bugs in new and existing
codebases.

\end{abstract}
