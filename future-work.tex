\section{Future Work}
\label{sec:future-work}

The ultimate goal of our future work on SEA and CrashSimulator is to build
a large, self-sustaining community of users around this technique and tool.
%To achieve this we have a
%plan that focuses on both improving the tool and making specific
%community building efforts.  In this section we detail what specifically
%needs to be done to accomplish both parts of this plan.
%
%
%In addition to new feature additions, CrashSimulator requires some updates to
%replace soon-to-be unsupported components.  This includes migration from
%Python 2 to Python 3.  We have migrated several widely used projects that have
%successfully migrated from Python 2 to 3, including TUF~\cite{TUFwebpage}.
%While this will take time, it does not pose substantial risk.
%\cappos{Python 2-3 is optional, cut if the space is needed.}
%
%dependency that requires updating is the processor architecture
%that CrashSimulator supports.  Because the tool performs its simulation
%work manipulating the memory and register values that contain the results
%and side effects of system calls, the initial version of the tool was
%constrained to only supporting x86, 32-bit systems in order to development
%time.  At this point, 64-bit systems (typically x86\_64) are the norm and
%many Linux distributions are reducing the effort put into their 32-bit
%flavors.  While it is possible to use CrashSimulator to test applications
%by re-compiling them for 32-bit x86, this reduces two of the tool's
%advantages -- the ``universal'' nature of its tests and its ability to test
%an application without its source code.  In order to avoid a cumbersome
%situation where is users must recompile their applications, CrashSimulator
%will be changed to support applications compiled for x86\_64.
%
%
%
We propose three areas of improvement in this work.  First, we
will work on enhancing the {\bf usability} of CrashSimulator, in both stand-alone
applications and in integrations into common toolchains.
This is needed
to make the tool more accessible to developers with differing levels of
expertise.
Second, we will address {\bf security} issues by creating
mechanisms that can safely execute mutators and checkers from
untrusted sources.
Third, we will build a website that will facilitate sharing mutators and
checkers with the larger {\bf community}.
We describe each of these in more detail below.


%\subsubsection{Tool Improvements and New Features}
%
%To make CrashSimulator more appealing to our target users we propose taking
%the following actions:


%JAC: This won't work.  You lose the meaning of the anomaly in most cases
% when moving to that level of abstraction.
%\subsubsection{Supporting new platforms}
%
%CrashSimulator has been shown to be successful at finding
%environmental bugs in Linux applications. While this is useful,
%there is a great deal of opportunity to find more interesting, impactful
%bugs by examining the environmental differences resulting from applications
%running on completely different operating systems.
%In order to find these
%bugs, CrashSimulator must be improved so that it can run on, and simulate
%anomalies present in, these additional platforms.  Initially we plan to
%support Windows and OS X.
%
%Key to realizing this support is dealing with the fact that each operating
%system provides different system calls.  For example, Linux uses {\tt
%open()} to open files while Windows provides {\tt NtOpenFile()}.  In their
%current form, checkers and mutators describe anomalies and expected
%application behavior in terms of Linux system calls.  Addressing this
%limitation means elevating this description to a higher level and
%implementing a layer to translate this description down to the actual
%system calls a given operating system provides.  Building upon the previous
%example, a higher level description in a mutator may specify an operation
%like ``open file'' which is translated to either {\tt open()} or {\tt
%NtOpenFile()} as is appropriate.
%
%An additional area for exploration is determining whether or not a given
%anomaly is able to be simulated on a given operating system.  If an anomaly
%only appears in the presence of a operating system feature , it would be an
%error to simulate it on operating systems where this feature is not
%present.  Classifying anomalies in terms of where it makes sense to
%simulate them would reduce the number of false positive reports the tool
%could make.


\paragraph{Improving Usability by Grouping and Prioritizing Anomalies}

Not every flaw found by CrashSimulator is equally important.  An issue that only
impacts an obsolete, rarely used file system is much less impactful than
one that makes the most popular file system vulnerable.
Anomalies
can be prioritized by grouping them into sets representing each
environment in which an application is expected to run. The more popular
the environment, the more the developer may want to be aware of the issue.
It may also be the case that developers want to prioritize fixing all
critical issues for one environment over another.  For example, a piece
of software used in the cloud may not need to worry about 
network-related issues that will only
occur behind the great firewall of China, if no cloud environments are deployed in this location.


%Grouping by environment and letting the developer prioritize and filter
%environments enables much better developer control and increases the useful
%true positive rate of CrashSimulator.
%For example, the corpus of anomalies included with a future version of CrashSimulator
%could be segregated into groups such as those present on Windows 10, those
%present on Ubuntu 18.04 and so on.  Once grouped in this fashion,
%applications could be tested against a chosen group in order to see how it
%would perform on the represented operating system without. In this way,
%CrashSimualtor could offer insights into how an application will perform on
%a given operating system without having to go through of the effort of
%deploying it.
%
%

\paragraph{Improving Usability via Integration into CI platforms}

As with other tools, CrashSimulator can only be effective if it is used.
Modern development processes include a continuous integration system,
such as TravisCI
and CircleCI, that run tests automatically.
CI systems are popular because they
ensure that applications are continually and thoroughly tested as new code
contributions are submitted.  Given CrashSimulator's ability to test
applications regardless of how they were constructed,
the tool is a perfect fit for
automatically testing numerous applications as part of a CI pipeline.
Our goal is to provide a configuration of CrashSimulator,
suitable for unattended execution, to which test jobs can be submitted
automatically as part of a continuous integration pipeline.  Once created,
this enhanced version of CrashSimulator would allow a user to test
applications against
a pre-selected set of anomalies by simply checking a box in the project's
testing configuration.

We believe this setup could drive rapid growth in the CrashSimulator
community because it removes most of the friction associated with adopting
a new tool.  Once the ``automated'' version of the tool is in place, it is a
natural next step for users to move into more advanced use cases.
With increased ease of use, these users will be more inclined
to contribute to the tool to meet their expanding needs.
Further, by
integrating with existing continuous integration infrastructure,
CrashSimulator gains access to those platforms' communities,
opening access to the tool to users who might not otherwise have found it.
The end result of all this could be a new, expanded, set of community
members that can contribute to the tool's continued improvement.

\paragraph{Facilitating Access by Building a Community Repository}

The community that grows around a tool or
technique is likely an even bigger contributor to its long term success
than its technical merits.  This is especially true in the case of SEA and
CrashSimulator since the effectiveness of the tool and technique
become more effective
as the corpus of anomalies grows.

One of the primary advantages offered by SEA and CrashSimulator is the way
their effectiveness and capabilities can be augmented by the addition of
new anomalies.  To best take advantage of this fact, it makes sense to
provide users as many anomalies as is feasible in an easily accessible
fashion.  In the simplest case, this involves supplying a set of
straightforward anomalies, ones known to be useful in testing a wide
variety of applications with the tool.
However, as the number of contributed anomalies grows, this approach
will not scale.  Shipping a large number of anomalies, including some
which will likely be useful only in specific situations, will harm the
tool's performance and user experience.  To address this problem, we
propose establishing an online repository
that can act as a centralized location for anomalies
and their related metadata,
such as the operating systems on which they appear,
their performance characteristics, and the sorts of applications
with which they are likely to be useful.  Additionally, it could also
act as a collection point for community-driven anomaly documentation
and discussion.  This information would allow users to
to sensibly download the individual anomalies or sets of anomalies that are most
useful for their specific use case.

Users will also be able to quickly like or dislike checkers and mutators,
and report instances of false positives to developers.
Modeled after the Python Package Index,
such a work can help the community find the most appropriate
anomalies for their particular applications.

The community repository will also have potential remediations linked to
the CrashSimulator tool.  For example, when a bug report about a
socket {\tt O\_NONBLOCK} flag inheritence is received, the user would be linked
to the Python
bug tracker's detailed report on the issue, along with the fix that
patches the bug.  This information helps developers fix the
bugs that CrashSimulator finds in a more timely manner.


\paragraph{Security via Sandboxing}

There are several reasons behind our desire to build a CrashSimulator community,
but one of the more important ones
is to build a corpus of anomalies for the tool to simulate.
In order to do this safely, the tool needs to be able to
take code from unknown and untrusted developers and execute it
without any negative effects to the host system.

Our approach to offering this capability is
to execute checkers and mutators, written in our description language in a
sandbox that isolates Python code.  The Python-Based Repy
sandbox~\cite{Cappos_CCS_2010} we propose to use
has already been tested across thousands of devices as a component of
the Seattle testbed~\cite{containmentseattle}.  The Repy sandbox
has been designed to execute a minimal subset of Python in a secure manner.
Additionally, it is constructed expressly to integrate into
existing programs like CrashSimulator.

The sandbox API will provide a
stream of system calls from the application under test
to the mutators and checkers along with the ability
to traverse backwards through some guaranteed number of prior
calls.  The mutator or checker may respond with either an ``{\tt accept}'' or
``{\tt reject}'' result based on this stream.  These limitations
provide strong protections to the host system by preventing
third-party checkers and mutators from
interacting with the host system in unintended ways.

Repy can also offer further protection by enforcing resource limitations on
the checkers and mutators running within it.
By imposing memory or processing limits, Repy can prevent
long running or poorly implemented checkers and mutators from harming the
stability of the host system.
As these
capabilities~\cite{Li_USENIX_2015} are already available through Repy,
the pairing will improve
not only the tool's ease of use, but also security.
These enhancements can help speed
community building by reducing the effort required to vet
community submissions.

%Furthermore, we believe the effort of actually writing a checker and mutator
%in the sandbox will be dramatically reduced in Repy.  Much of the
%complexity associated with constructing these elements currently comes down
%to their being implemented in fully featured python.  Repy has demonstrated
%a remarkable ability to make writing many programs simpler by having a clean
%API and removing complexity from the language itself.
%%could be reduced through the use of a simpler checker and mutator
%%description language description language.  As we have discussed in more
%%detail elsewhere in this document, our plans consists of a language akin to
%%regular expressions focusing on sequences of system calls and their
%%arguments rather than the contents of strings.
%Proof of concept
%experiments with this design have shown promise both in terms of
%feasibility of implementation and in real benefits to the anomaly encoding
%process.




%\subsubsection{Usability Improvements}
%\cappos{2}
%
%Other improvements we propose will directly benefit CrashSimulator's
%usability.  The primary developer effort associated with using
%CrashSimulator is in constructing the checkers and mutators that drive its
%testing process.  In much the same vein as a constructing a unit test, the
%analysis effort required to produce these artifacts is closely tied to an
%individual user's skill and difficult to improve through technical means.


%\subsubsection{Updating components}



%\subsubsection{Giving New Features Back to the Community}
%
%Part of the effort of developing CrashSimulator was making modifications to
%the {\tt rr} record-and-replay debugger.  Specifically, these changes
%allowed for two new features: the output of strace-style recordings and the
%ability to clone process sets as discussed in the SEA paper.  We propose
%giving these features back to the {\tt rr} community by porting them
%into main-line {\tt rr}.  Doing so would benefit {\tt rr}'s
%users by allowing them access to these new capabilities and would help
%CrashSimulator by ensuring that functionality on which it depends will
%continue to exist as {\tt rr} continues to grow and improve.


%\end{document}


%  Documentation on identifying and capturing anomalies
%  Product evangelism
%  User study
%  Wider instruction on selecting and capturing anomalies
